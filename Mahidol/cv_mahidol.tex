%%%%%%%%%%%%%%%%%%%%%%%%%%%%%%%%%%%%%%%%%
% Medium Length Graduate Curriculum Vitae
% LaTeX Template
% Version 1.1 (9/12/12)
%
% This template has been downloaded from:
% http://www.LaTeXTemplates.com
%
% Original author:
% Rensselaer Polytechnic Institute (http://www.rpi.edu/dept/arc/training/latex/resumes/)
%
% Important note:
% This template requires the res.cls file to be in the same directory as the
% .tex file. The res.cls file provides the resume style used for structuring the
% document.
%
%%%%%%%%%%%%%%%%%%%%%%%%%%%%%%%%%%%%%%%%%

%----------------------------------------------------------------------------------------
%	PACKAGES AND OTHER DOCUMENT CONFIGURATIONS
%----------------------------------------------------------------------------------------

\documentclass[margin, 10pt]{res} % Use the res.cls style, the font size can be changed to 11pt or 12pt here

\usepackage{helvet} % Default font is the helvetica postscript font
%\usepackage{newcent} % To change the default font to the new century schoolbook postscript font uncomment this line and comment the one above

\setlength{\textwidth}{5.1in} % Text width of the document

\begin{document}

%----------------------------------------------------------------------------------------
%	NAME AND ADDRESS SECTION
%----------------------------------------------------------------------------------------

\moveleft.5\hoffset\centerline{\large\bf CHAIYONG RAGKHITWETSAGUL} % Your name at the top
 
\moveleft\hoffset\vbox{\hrule width\resumewidth height 1pt}\smallskip % Horizontal line after name; adjust line thickness by changing the '1pt'
 
\moveleft.5\hoffset\centerline{Faculty of Information and Communication Technology} % Your address
\moveleft.5\hoffset\centerline{Mahidol University, Nakhon Pathom, Thailand}
\moveleft.5\hoffset\centerline{Email: chaiyong.rag@mahidol.ac.th, cragkhit@gmail.com}
\moveleft.5\hoffset\centerline{Personal Website: http://cragkhit.github.io}

%----------------------------------------------------------------------------------------

\begin{resume}

%----------------------------------------------------------------------------------------
%	OBJECTIVE SECTION
%----------------------------------------------------------------------------------------
 
%\section{OBJECTIVE}  

%US VISA application.

%----------------------------------------------------------------------------------------
%	PERSONAL PROFILE SECTION
%----------------------------------------------------------------------------------------

\section{PERSONAL PROFILE}  

I am a lecturer at the Faculty of Information and Communication Technology, Mahidol University in Thailand. I was a PhD student under the supervision of Dr.~Jens Krinke and Dr.~David Clark in the Centre for Research on Evolution, Search and Testing (CREST), Department of Computer Science, University College London, and successfully defended my thesis on 30\textsuperscript{} August 2018. 
%I am also an assistant instructor at the Faculty of ICT, Mahidol University who is the main sponsor for my PhD study.

%----------------------------------------------------------------------------------------
%	CURRENT PROJECTS SECTION
%----------------------------------------------------------------------------------------

%\section{CURRENT PROJECTS}  

%CloneNets: Detecting code clones in JavaScript using recurrent neural networks. \vspace{1mm} \\
%Siamese: An incremental and scalable clone search engine via multiple code representations \vspace{1mm} \\
%Cloverflow: An empirical study of toxic code snippets on Stack Overflow \vspace{1mm} \\
%A comparison of code similarity analysers \vspace{1mm} \\
%Optimising configurations of code clone detection tools

%----------------------------------------------------------------------------------------
%	PUBLICATION SECTION
%----------------------------------------------------------------------------------------

\section{PUBLICATIONS}  

M. Paixao, J. Krinke, D. Han, C. Ragkhitwetsagul, M. Harman. The Impact of Code Review on Architectural Changes. Transactions on Software Engineering (TSE), 2019.
\vspace{2mm} \\
C. Ragkhitwetsagul, J. Krinke. Siamese: Scalable and Incremental Code Clone Search via Multiple Code Representations. Empirical Software Engineering (EMSE), 2019. 
\vspace{2mm} \\
C. Ragkhitwetsagul, J. Krinke, M. Paixao, G. Bianco, R. Oliveto, \textit{Toxic Code Snippets on Stack Overflow}, IEEE Transactions on Software Engineering, 2019.
\vspace{2mm} \\
J. Wilkie , Z. Al Halabi , A. Karaoglu , J. Liao , G. Ndungu, C. Ragkhitwetsagul, M. Paixão , J. Krinke (2018). \textit{Who's this? Developer identification using IDE event data}. In the 15th International Conference on Mining Software Repositories -- Mining Challenge (MSR 2018), 2018. Gothenburg, Sweden 
\vspace{2mm} \\
C. Ragkhitwetsagul, J. Krinke (2018). \textit{A picture is worth a thousand words: code clone detection based on image similarity.} In the 12th International Workshop on Software Clones (IWSC 2018), 2018. Campobasso, Italy -- The People's Choice Award\vspace{2mm} \\
C. Ragkhitwetsagul, J. Krinke, R. Oliveto (2017). \textit{Awareness and Experience of Developers to Outdated and License-Violating Code on Stack Overflow: An Online Survey.} UCL Computer Science Research Note (RN/17/10), 2017. \vspace{2mm} \\
C. Ragkhitwetsagul, J. Krinke, D.Clark (2017). \textit{A Comparison of Code Similarity Analysers.} Empirical Software Engineering, https://doi.org/10.1007/s10664-017-9564-7, 2017. \vspace{2mm} \\
M. Paixao, J. Krinke, D. Han, C. Ragkhitwetsagul and M. Harman (2017). \textit{Are Developers Aware of the Architectural Impact of Their Changes?.} In the 32nd IEEE/ACM International Conference on Automated Software Engineering (ASE 2017), 2017. Illinois, USA \vspace{2mm} \\
C. Ragkhitwetsagul, J. Krinke (2017). \textit{Using Compilation/Decompilation to Enhance Clone Detection.} In 11th International Workshop on Software Clones, 2017. Klagenfurt, Austria -- The People's Choice Award \vspace{2mm} \\
C.~Ragkhitwetsagul, M.~Paixao, M.~Adham, S.~Busari, J.~Krinke, and J.~H.~Drake (2016). \textit{Searching for Configurations in Clone Evalution: A Replication Study.} In 8th International Symposium on Search-based Software Engineering (SSBSE): Challenge Track, 2016. Raleigh, NC, USA. (NSF Student Travel Support) \vspace{2mm} \\
C. Ragkhitwetsagul, J. Krinke, D. Clark (2016). \textit{Similarity of Source Code in the Presence of Pervasive Modifications.} In 16th IEEE International Working Conference on Source Code Analysis and Manipulation (SCAM), 2016. Raleigh, NC, USA. \vspace{2mm} \\
C. Ragkhitwetsagul (2016). \textit{Measuring Code Similarity in Large-scaled Code Corpora.} In 32nd International Conference on Software Maintenance and Evolution (ICSME): Doctoral Symposium, 2016. Raleigh, NC, USA. 
\vspace{2mm} \\
P. Janviriya, T. Ongarjithichai, P. Numruktrakul, C. Ragkhitwetsagul. (2014). \textit{CloudyDays : Cloud Storage Integration System.} In ICT-ISPC 2014 (pp. 125-128). Nakhonpathom, Thailand. 
\vspace{2mm} \\
P. Hathaiwichian, L. Siriwittayacharoen, A. Wongwachirawanich, C. Ragkhitwetsagul (2014). \textit{Android Application for Event Management and Information Propagation.} In ICT-ISPC 2014 (pp. 139-142). Nakhonpathom, Thailand.

%----------------------------------------------------------------------------------------
%	EDUCATION SECTION
%----------------------------------------------------------------------------------------

\section{EDUCATION}
\textbf{2014 -- 2018, PhD}\\
\textbf{PhD in Computer Science}\\
University College London, UK\\
Research topic:  Code similarity and clone search in large-scale source code data \\
Supervisors: Dr. Jens Krinke, Dr. David Clark\\
Research interest: code clone/plagiarism detection, code similarity, mining of software repositories, search-based software engineering 

\textbf{2007 -- 2008, Master} \\
\textbf{Master of Science in IT -- Very Large Information Systems}\\
Carnegie Mellon University, USA\\
Capstone project: Honeydew -- Meeting date prediction using machine learning \vspace{3mm} \\
\textbf{2002 -- 2005, Undergraduate }\\
\textbf{Bachelor Degree in Computer Engineering (Magna cum Laude)}\\
Kasetsart University, Thailand

%----------------------------------------------------------------------------------------
%	AWARDS SECTION
%----------------------------------------------------------------------------------------

\section{AWARDS}
Industry Academia Partnership Programme 18-19 (13,345 GBP) \hfill 2019 \\ 
Royal Academy of Engineering, UK (Newton Fund) \vspace{2mm} \\
\textit{Automated Software Engineering for Thailand Software Industry} \vspace{2mm} \\
Amazon AWS Credits for Research (20,000 USD) \hfill 2018 \\ 
\textit{Code Clone Search as a Web Service} \vspace{2mm} \\
The People's Choice Award \hfill 2018 \\ 
\textit{A picture is worth a thousand words: code clone detection based on image similarity} \\
\textit{IWSC 2018. Campobasso, Italy.} \vspace{2mm} \\
The People's Choice Award \hfill 2017 \\ 
\textit{Using Compilation/Decompilation to Enhance Clone Detection} \\
\textit{IWSC 2017. Klagenfurt, Austria.} \vspace{2mm} \\
NSF Student Travel Support (700 USD) \hfill 2016 \\ 
\textit{SSBSE 2016. Raleigh, NC, USA.} \vspace{2mm} \\
Microsoft Azure Research Award (20,000 USD) \hfill 2016 \\ 
\textit{ISiCS: Internet-scaled Similar Code Search project} \vspace{2mm} \\
Full PhD scholarship \hfill 2014 -2018 \\
\textit{Mahidol University Academic Development Scholarship \\ 
\& Faculty of ICT, Mahidol University Scholarship} \vspace{2mm} \\
Best Project Award: 15-637 Web App Development,\hfill 2008 \\Carnegie Mellon University \\ 
\textit{WhimsyWord: multi-player online hangman game} \vspace{2mm} \\
Full Master degree scholarship \hfill 2007 - 2008 \\
\textit{The Royal Thai Government scholarship} \vspace{2mm} \\
Winner of the 8th National Software Contest of Thailand (NSC) \hfill 2006 \\ 
Category -- Software for Entertainment \\
\textit{Sword Edge: virtual reality fighting game} \vspace{2mm} \\
DTAC \& NOKIA i-Awards mobile app contest: finalists \hfill 2005  \\
\textit{Interactive e-cards mobile application}
 
%----------------------------------------------------------------------------------------
%	PROFESSIONAL EXPERIENCE SECTION
%----------------------------------------------------------------------------------------
 
\section{EXPERIENCE}

{\sl Research Student} \hfill Sep 2014 -- Sep 2018 \\
\textbf{Centre for Research on Evolution, Search and Testing (CREST), \\ Software Systems Engineering (SSE) group, \\ Department of Computer Science, University College London}\\
Conducting research in code similarity, code reuse, code clone/plagiarism detection, and search-based software engineering \vspace{4mm} \\
{\sl Research Intern} \hfill May 2017 -- Jul 2018 \\
\textbf{Prodo.ai, London}\\
Conducting research in deep learning for code clone detection 
\vspace{4mm} \\
{\sl Assistant Instructor} \hfill Dec 2012 -- Sep 2014 \\
\textbf{Faculty of ICT, Mahidol University, Nakhonpathom, Thailand} \\
Lectured Fundamentals of Programming, OOP, DBMS, Knowledge-based Systems, and Mobile App Development. \\
Advised three undergraduate student projects. \vspace{3mm} \\
{\sl Consultant} \hfill Nov 2011 - Sep 2014 \\
\textbf{Blaccess Thailand} \\
Made key decisions and providing company's vision. \\
Planned and managed IT projects and staffs. \vspace{3mm} \\
{\sl System Analyst} \hfill Jan 2009 - Nov 2012\\
\textbf{National Economic and Social Development Board, Thailand}\\
Planned, maintained, and developed NESDB IT services (e.g.~email systems, web applications, DBMS).\\
Administrated NESDB applications and computer networks. \vspace{3mm} \\
{\sl Teacher} \hfill Feb 2010 - Dec 2012 \\
\textbf{ProGaming, Thailand}\\
Instructed Pro Android course on developing Android applications.\\
Instructed Pro iOS course on developing iOS applications. \vspace{3mm} \\
{\sl Developer} \hfill Jan 2009 - Dec 2012 \\
\textbf{Kasetsart University, Thailand}\\
Developed a digital repository, and training/project tracking system for Puparn Royal Development Study Center. \vspace{3mm} \\
{\sl Intern} \hfill Mar 2008 - Aug 2008 \\
\textbf{Oracle, USA}\\
Developed of a new version of Oracle's internal testing platform, Distributed  
Testing System (DTS) 2.0, using Java web technology and Javascript. \vspace{3mm} \\
{\sl Software Test Engineer} \hfill Feb 2006 - Apr 2007\\
\textbf{Microsoft, Thailand}\\
Created automated test tools for testing localized versions of Windows Vista User Assistance Platform.\\
Tested user assistance (UA) contents in XML format.

%----------------------------------------------------------------------------------------
%	SKILLS SERVICE SECTION
%---------------------------------------------------------------------------------------- 

%\section{SKILLS}
%\textbf{Platforms: } macOS, Linux, Windows, Raspberry Pi, iOS, Android \\
%\textbf{Programming languages: } Python, Shell scripts, Java, C/C++, Obj-C, SQL

%----------------------------------------------------------------------------------------
%	PROFESSIONAL ACTIVITIES SECTION
%----------------------------------------------------------------------------------------

\section{PROFESSIONAL \\ ACTIVITIES} 
Program Committee:~IWSC 2019, ICPC 2019, ICSME 2019 \hfill 2019\\
Publicity Chair:~IWSC 2018 \hfill Mar 2018 \\
Reviewer:~EMSE (2019), IEEE Access (2019), ESWA (2018), JSS (Nov 2017), IET (Oct 2017), STVR (Apr 2017), PeerJ Computer Science (May 2016) \\
Subreviewer:~ICSE '18, MSR '18, REVE'17, ICSME'17, MSR'17, IWSC'17, SANER'17, SCAM'16, SCAM'15

%----------------------------------------------------------------------------------------

%----------------------------------------------------------------------------------------
%	REFERENCES SECTION
%----------------------------------------------------------------------------------------

\section{REFERENCES} 

References upon request.
%----------------------------------------------------------------------------------------

\end{resume}
\end{document}